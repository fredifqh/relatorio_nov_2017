\documentclass[a4paper, 11pt]{article}
\usepackage{graphicx}
\usepackage{fourier}
\usepackage{anysize}
\usepackage{amsmath}
\usepackage{booktabs}          %Para ter mais opcoes nas tabelas
\usepackage[brazilian]{babel}
\usepackage[utf8]{inputenc}
\usepackage{float}
%\usepackage{pdfpages}
\usepackage{verbatim}
\usepackage[colorlinks]{hyperref}
\usepackage{multirow}
%\usepackage[comma]{natbib}
\usepackage{comment}
\usepackage[compress,comma,]{natbib}
\usepackage[bottom]{footmisc}    %footnote in bottom position
\usepackage{enumitem}
\marginsize{2.5cm}{2cm}{2cm}{2cm}

\renewcommand{\baselinestretch}{1.2}

\begin{document}
\renewcommand{\figurename}{\textsc{Figura}}
\renewcommand{\tablename}{\textsc{Tabela}}
\renewcommand{\refname}{Refer\^encias}

\numberwithin{table}{section}

\thispagestyle{empty}
{\large

{\scshape Observat\'orio Nacional

Minist\'erio da Ci\^encia, Tecnologia e Inova\c{c}\~ao - MCTI

Programa de P\'os-gradua\c{c}\~ao em astronomia}

\vspace{1cm}

\begin{center}
\includegraphics[width=5cm]{logoON.jpg} 
\end{center}

\vspace{1cm}

\begin{center}
\bfseries\Large Astrometria de TNOs e Centauros na era Gaia: procedimentos observacionais e imagens de época antiga
\end{center}

\vspace{1cm}

{\scshape
\begin{center}
Aluno: Fredi Quispe Huaynasi	

Orientador: Dr. Julio Ign\'acio Bueno de Camargo

Co-orientador: Dr. Gustavo Beneddeti Rossi
\end{center}

\vspace{1cm}

\begin{center}
Inicio de Mestrado: Setembro de 2016 

Bolsista da CNPq: Setembro de 2016
\end{center}}

\vfill

\begin{center}
Rio de Janeiro

Novembro de 2017 
\end{center}



}

\newpage


\section{Projeto}

\subsection{Objetivo científico}

A importância do estudo de pequenos corpos al\'em da \'orbita de Netuno (transnetunianos ou TNOs, na sigla em ingl\^es) est\'a no fato de que estes s\~ao objetos que melhor conservaram em si caracter\'{\i}sticas do Sistema Solar primordial. Em outras palavras, devido às suas grandes dist\^ancias do Sol, eles s\~ao remanescentes pouco alterados da forma\c c\~ao do Sistema Solar.
% * <gugabrossi@gmail.com> 2017-11-24T12:05:21.505Z:
% 
% Sistema Solar é sempre escrito com as primeiras letras escritas em maiúsculo: "Sistema Solar". Troque todos no texto...
% 
% "por conta de suas" => "devido às suas"
% 
% ^.

Centauros s\~ao objetos em \'orbitas inst\'aveis entre J\'upiter e Netuno, com prov\'avel origem na regi\~ao transnetuniana. Por isso, e por suas menores dist\^ancias helioc\^entricas quando comparadas \`as dos TNOs, os Centauros s\~ao considerados similares, mas com maior brilho dos transnetunianos. Assim, s\~ao tamb\'em relavantes em estudos ligados \`a origem e forma\c c\~ao do Sistema Solar externo.
% * <gugabrossi@gmail.com> 2017-11-24T12:09:56.909Z:
% 
% "mais" = use para adição ; "mas" =  use para apresentar uma ideia oposta
% considerados similares, MAS com maior brilho...
% 
% ^.

Uma forma de se estudar tais objetos \'e atrav\'es de oculta\c c\~oes estelares, ou seja, da an\'alise da curva de luz obtida da oculta\c c\~ao de uma estrela por um TNO ou Centauro. Certamente, a predi\c c\~ao de eventos de oculta\c c\~ao \'e um passo essencial nesse estudo. Neste contexto, e tamb\'em visando o grande n\'umero de objetos al\'em da \'orbita de Netuno que ser\~ao descobertos por grandes levantamentos, os objetivos deste trabalho s\~ao: astrometria de imagens diretas a partir do solo de TNOs e Centauros usando o cat\'alogo Gaia; manuten\c c\~ao de base de dados com predi\c c\~oes de oculta\c c\~oes estelares para os TNOs e Centauros conhecidos; determina\c c\~ao de uma posi\c c\~ao do Centauro Chariklo atrav\'es da an\'alise de curva de luz oriunda da  oculta\c c\~ao estelar ocorrida em 24/AGO/2017.
% * <gugabrossi@gmail.com> 2017-11-24T12:12:15.612Z:
% 
% Essential => essencial
% 
% ^.
\subsection{Estado da arte}

A técnica de ocultação estelar por TNOs e Centauros, complementada por outras técnicas, permite obter o tamanho destes corpos na ordem do quilômetro, caracterizar suas formas, e revelar propiedades interessantes como anéis (\cite{2017Natur.550..219O, 2014AGUFM.P43F..01B}) e atmosferas em torno de eles (\cite{1538-3881-136-5-1757, 2006Natur.439...52S}). No entanto, devido ao pequeno tamanho dos TNOs e Centauros (menores que 2400 km de diâmetro) e dada suas dist\^ancias a partir do Sol (entre  5 - 30 ua para os Centauros, e maior que 30 ua para os TNOs), possuem baixo brilho (magnitudes tipicamente maiores que V=20), a determina\c c\~ao da posi\c c\~ao desses objetos, e como consequ\^encia a previsão de ocultações por eles não é uma tarefa simples. 
% * <gugabrossi@gmail.com> 2017-11-24T12:16:30.224Z:
% 
% * de esses = destes
% * caracterizar suaS formaS...
% * propriedades muito interessantes = > propriedades interessantes 
% *Verifique como citar a bibliografia... o ano deve estar em parênteses ou a separação entre as citações deve ser trocada por ; 
% * devido que os TNOs e Centauros são pequenos => devido ao pequeno tamanho dos TNOs e Centauros
% * (menores que 2400 km DE DIÂMETRO)
% * Centauros sempre deve ter a primeira letra maiúscula! Verifique no texto todo!
% * 30 ua para os TNOs), possuem...
% * a determinação da posição desses objetos e como consequência a previsão de ocultações por eles não
% 
% 
% ^.
 
 Esforços observacionais e de tratamento de dados, como por exemplo aqueles apresentados por \citet{2010A&A...515A..32A} e \citet{2014A&A...561A..37C}, eram voltados a refinar posições estelare e dos TNOs/Centauros mais brilhantes (V$\leq$21) com o intuito de obter predições de ocultações estelares com maior precisão. Este "com maior precisão"  é dado em comparação à utilização direta de informações contidas em catálogos estelares (como o UCAC4, por exemplo) e em efemérides de pequenos corpos (como aquelas oriundas do JPL). Tais esfor\c cos colaboraram para um crescente número não apenas no número de ocultações observadas mas também na quantidade de objetos envolvidos. De todo o modo, predições ainda constituem uma das etapas mais sensíveis do estudo de pequenos corpos distantes no Sistema Solar através de ocultações estelares. 
% * <gugabrossi@gmail.com> 2017-11-24T12:25:44.444Z:
% 
% * A citação de autores diretos no texto não pode ter ( )... o comando no latex pra isso é o "citet{}". Também você deve acrescentar um 'e' entre os dois autores (ou se tiver mais, um 'e' antes do último autor
% 
% ^.

Atualmente, no início da era Gaia e dos grandes levantamentos, estamos numa época da transição. A missão espacial Gaia irá fornecer posições com precisões abaixo do milésimo do segundo (veja \citet{2017arXiv171010816K} para ver os resultados astrométricos oferecidos pelo Gaia a partir de abril de 2018) e, virtualmente, sem erros sistemáticos. Assim, posições estelares não serão mais um problema nas predições. Efemérides precisas para os pequenos corpos, por outro lado, requerem continuidade dos esforços mas num ambiente bastante mais favorável.   
% * <gugabrossi@gmail.com> 2017-11-24T12:29:31.603Z:
% 
% * Se 'atualmente' é a era gaia e dos grandes levantamentos, não estamos numa época de transição... você pode mudar para:
% 'Atualmente, no início da era Gaia..."
% 
% * referência citada direta
% 
% ^.

Uma nova redução astrométrica de imagens históricas desses corpos, feitas com o Gaia como referência, será de grande valor. De fato, dados os longos períodos que esses corpos possuem, posições antigas possuem um peso importante na determinação de boas órbitas. A boa manutenção dessas órbitas, bem como a descoberta de dezenas de milhares de outros objetos, será obtida através de grandes levantamentos como, por exemplo, o {\it Large Synoptic Survey Telescope} (LSST) (\citet{2009arXiv0912.0201L}).

Este trabalho, como mencionado anteriormente, está engajado nesse cenário através da redução de imagens de TNOs e Centauros com o catálogo Gaia e na manutenção de um banco predições para o maior número possível de, inicialmente, TNOs e Centauros.

\newpage
\subsection{Metodologia}

A metodologia para conseguir nosso objetivo será dividida em três partes:
% * <gugabrossi@gmail.com> 2017-11-24T12:37:36.223Z:
% 
% Acho que aqui você pode listar as 3 partes e depois explicar abaixo o que é cada uma delas...
% 
% ^.

\begin{itemize}
  \item Metodologia para obter a astrometria usando imagens.
  \item Metodologia para obter a astrometria através de ocultações estelares.
  \item Metodologia para a manuten\c c\~ao e atualiza\c c\~ao da base de dados de predi\c c\~oes.
\end{itemize}

\noindent Cada uma das metodologias são detalhadas em seguida.
\subsubsection{Metodologia para obter a astrometria usando imagens}
\begin{enumerate}
  \item Revisitar a base de imagens recentes (2016) e antigas (2012 e 2013) dos TNOs e Centauros que foram obtidas com a câmera WFI (Wide Field Manager) instalada no telescópio de 2,2m do ESO.
  \item Reduzir as imagens (bias, flat ) utilizando o pacote esowfi no IRAF (Image Reduction And Analysis Facility). 
  \item Determinar a astrometria dos corpos utilizando o pacote PRAIA (Plataforma para Redução Automática de Imagens Astronômicas) usando os dados do catalogo GAIA (no momento, o Gaia {\it Data Release 1} (\citet{2016A&A...595A...4L}) e, assim que disponível, o Gaia {\it Data Release 2} (\citet{2017arXiv171010816K}).
\end{enumerate}
% * <gugabrossi@gmail.com> 2017-11-24T12:39:08.051Z:
% 
% PRAIA serve a sigla em português também: Plataforma para Redução Automática de Imagens Astronômicas
% 
% ^.

\subsubsection{Metodologia para obter a astrometria através de ocultações estelares}
\begin{enumerate}
  \item Analisar as curvas de luz de ocultações estelares observadas para obter os tempos de imersão e emersão (cordas).
  \item Ajustar a forma do corpo às cordas.
  \item Determinar a posição do centro do objeto.
  \item Obter a posição relativa da posição do centro do objeto com a estrela ocultada.
\end{enumerate}

\subsubsection{Metodologia para a manuten\c c\~ao e atualiza\c c\~ao da base de dados de predi\c c\~oes}
\begin{enumerate}
  \item Buscar no Johnston's Archive (\url{http://www.johnstonsarchive.net/astro/tnoslist.html}) a lista dada de TNOs e Centauros conhecidos
  \item Baixar arquivos SPK\footnote{Arquivos que cont\'em dados de efem\'erides de objetos do Sistema Solar.} de todos esses corpos a partir do JPL. Para isso, \textit{scritp} desenvolvido pelo JPL \'e utilizado.
  \item A partir desses arquivos, utiliza-se c\'odigo baseado em rotinas do sistema SPICE para se gerar efem\'erides geoc\^entricas, em passos de 10 minutos, para todos esses objetos para um dado ano (por exemplo, 2018).
  \item Obt\'em-se estrelas Gaia nas vizinhan\c cas de cada \'orbita anual.
  \item Cruza-se as posi\c c\~oes de efem\'erides com as respectivas posi\c c\~oes estelares nas vizinhan\c cas das \'orbitas.
  \item Gera-se os mapas de oculta\c c\~ao.
\end{enumerate}

\subsection{Resultados esperados}

Das metodologias eu espero ter o seguintes resultados:
\begin{itemize}
  \item Da primeira, eu espero ter obtido a posição de 40 TNOs e 6 Centauros com precisão entre 5 e 30 $mas$ usando a segunda liberação de dados de GAIA.
  \item Da segunda, eu espero dominar o processo para se obter posições a partir ocultações estelares. Neste trabalho pretendo determinar a posição de Chariklo usando a ocultação observada o dia 24 de Agosto de 2017, com uma precisão na ordem de $mas$. 
  \item Da última, eu espero desenvolver o procedimento automático para se atualizar a base de dados de previsão de ocultações do site,  \url{http://www.linea.gov.br/020-data-center/acesso-a-dados-3/tno-dados/} duas
  vezes ao ano. Essa atualização tem em conta não apenas a atualização das efemérides mas também a inclusão de objetos novos. O intuito inicial do base de dados não é fornecer predições com grande precisão mais sim um material seguro para escolhermos eventos sobre os quais maior atenção será dedicada.
\end{itemize}
% * <gugabrossi@gmail.com> 2017-11-24T12:42:41.760Z:
%
% ^.


\section{Informa\c{c}\~oes curriculares}

\subsection{Disciplinas cursadas}

\begin{center}
\scalebox{0.8}{
\begin{tabular}{clccc}
\toprule
\textbf{N}&$\hspace{2cm}$\textbf{Disciplina} &\textbf{Ano/Semestre}&\textbf{Conceito}& \textbf{Cr\'editos} \\ \midrule
1 & Astronomia observacional                                             & 2016/2 & A & 04 \\ 
2 & Astroestatística                                                     & 2016/2 & A & 02 \\ 
3 & Astrofísica Planetária                                               & 2016/2 & B & 04 \\ 
4 & Tópicos em Formação e Evolução Primordial do Sistema Solar           & 2016/2 & B & 02 \\
5 & Astronomia de posição                                                & 2017/1 & A & 04 \\ 
6 & Astronomia Dinâmica                                                  & 2017/1 & A & 04 \\ 
7 & Métodos Númericos em Astronomia Dinâmica                             & 2017/1 & A & 04 \\
8 & Tópicos Avançados de Dinamica Planetaria                             & 2017/2 & Andamento & \\
9 & Tópicos de ocultações no Sistema Solar                                & 2017/2 & Andamento & \\
 \midrule 
\multicolumn{4}{c}{\textbf{Total de cr\'editos obtidos}}               & \textbf{24} \\ \bottomrule
\end{tabular}}
\end{center}

 Os créditos necessários para o mestrado ja foram completadas.

\newpage
\subsection{Cursos e reuni\~oes cient\'ificas}

\begin{itemize}
  \item Participação no XXI Ciclo de Cursos Especiais ON/MCT, 2016, nos seguintes temas: 

  Computational Cosmology: Simulating Cosmic Structures, Prof. August Evrard (University of Michigan, US); \\
  Comets: Origin, Evolution and Interrelations, Prof. Julio Fernández (University of the Republic, Uruguay); \\
  Advanced stellar evolution: basic properties, nucleosynthesis, and final fates, Prof. Paola Marigo (University of Padova, Italia).

  \item Participação na Second Astrobiology school at Observatório Nacional. 2016, realizado no Rio de Janeiro, ON-MCTI, nos seguintes temas: 

  Star Formation and Planet Formation and Habitability: From the Solar neighborhood to the Multiverse, Prof. Fred C. Adams (University of Michigan, USA); \\ The Pathway to Earth 2.0: Discovery and Characterization of the Nearest Exo-Earths, Prof. James S. Jenkins (Universidad de Chile); \\
  Dust in Protoplanetary Disks: The first Steps of Planet Formation, Prof. Isa Oliveira (Observatorio Nacional, Brasil); \\
  Degradation of amino acids by cosmic radiation, Prof. Enio Frota da silveira; \\
  Kepler Mission: Planet Candidates Characteristics and Earth Analogs Ocurrences Rates, Prof. Eduardo Seperuelo Duarte (UFRJ); \\
  Energetic Processing of Materials Relevant to Astrobiology, Prof. Daniele Fulvio (PUC do Rio de Janeiro)
  

  \item Participação no XXII Ciclo de Cursos Especiais ON/MCT, 2017, nos seguintes temas: 

  Extrasolar Planets Around Nearby Stars, Dr. R. Paul Butler (Carnegie Institution of Washington, EUA); \\
  Observing the Formation and Evolution of Galaxies over 13 Billion Years, Dr. Christopher Conselice (University of Nottingham, Reino Unido); \\
  Impacts in the Solar System, Dr. Patrick Michel (Observatoire de la Cote d' Azur, França); \\
  The Theory and Applications of the Baryon Acoustic Oscillations, Dr. Nikhil Padmanabhan (Yale University, EUA).
\end{itemize}

\subsection{Exame de profici\^encia em L\'ingua Inglesa}
O exame de proficiência será feita no dia 5 de Abril de 2018.

\newpage
\subsection{Outras atividades}

Alem das atividades para cumprir com o objetivo cientifico, participei das seguintes atividades:

\begin{enumerate}
  \item Participação na construção do site \url{http://www.linea.gov.br/020-data-center/acesso-a-dados-3/tno-dados/}, a qual fornece predição de ocultações para TNOs e Centauros. 
  \item Participação em campanhas de observação remota nos telescópios do observatório Picos dos Dias. 
% * <gugabrossi@gmail.com> 2017-11-24T12:46:13.028Z:
% 
% As observações foram todas remotas, certo?
% 
% ^.
  \begin{itemize}
    \item Julho (noite 27 e 28)
    \item Agosto (noite 9 e 23) % 
    \item Setembro (noites 19 e 20)
    \item Outubro (noites 21 e 22)
  \end{itemize}
\end{enumerate}
%==============================================================
\section{Situação do projeto de pesquisa à época do último relatório}
%==============================================================
Este é o meu primeiro relatório como aluno da DPPG.
%============================================================================
\section{Descrição detalhada do trabalho de pesquisa desenvolvido no período}
%============================================================================

%===================================================================
\subsection{Metodologia aplicada}
%===================================================================
Para obter a astrometria de TNOs e Centauros, at\'e o momento, foram reduzidas 1328 imagens obtidas no telesc\'opio de 2.2m do ESO com a c\^amera WFI durante os anos 2012 e 2013. As reduções (Bias e Flat) foram feitas usando a ferramenta \textit{esowfi} no IRAF. As imagens reduzidas correspondem aos objetos mostradas na primeira coluna da tabela \ref{table:incerteza}. Para obter a posição dos corpos as imagens foram fornecidas à ferramenta PRAIA usando os dados do Gaia DR1 como refer\^encia astrom\'etrica. 

Para a manutenção e atualização da base de dados de ocultações estelares hospedadas em \url{http://www.linea.gov.br/020-data-center/acesso-a-dados-3/tno-dados/}, {\it scripts} em {\it bash} para as etapas de determinação das efemérides dos corpos, extração de posições estelares Gaia ao redor das órbitas desses corpos, e determinação dos mapas de predição estão sendo integrados para alcançarmos um procedimento automático de atualização da base de predições. Individualmente, estes {\it scripts} já existem. Em particular, escrevi uma rotina para obter de forma automática a informação do diâmetro do corpo usando a tabela con informação dos TNOs fornecida no site \url{http://www.johnstonsarchive.net/astro/tnoslist.html}. Este \textit{script}  é um passo importante para automatizar o processo para a predição de ocultações estelares. Em breve, essa melhoria constará nos mapas contidos na URL acima.

%===================================================================
\subsection{Resultados parciais já obtidos}
%===================================================================

Foram determinadas a astrometria de 43 TNOs usando as imagens obtidas durante os anos 2012 e 2013 pelo telescópio 2,2 m do ESO, e a primeira liberação de dados (Gaia DR1). A tabela \ref{table:incerteza} mostra os resultados já obtidos.  

\begin{table}
\begin{center}
\caption{Incertezas associadas à posição dos TNO e Centauros ($\sigma_{\alpha} \cos \delta, \sigma_{\delta}$) em segundos de arco.}
\vspace{0.3 cm}
\scalebox{0.8}{
\begin{tabular}{|l|l|c|c|} 
\hline
\hline 
ID          & Classe &  $\sigma_{\alpha} \cos \delta $& $\sigma_{\delta} $        \\
\hline
1995GO (Absolus)   &  Cen  &  0.035 & 0.017 \\ 
1995SM55    &  TNO &  0.020 & 0.029 \\ 
1999TC36    &  TNO &  0.006 & 0.007 \\ 
2000QC243 (Bienor) &  Cen &  0.004 & 0.004 \\ 
2002AW197   &  TNO &  0.012 & 0.015 \\ 
2002TC302   &  TNO &  0.028 & 0.014 \\ 
2002UX25    &  TNO &  0.009 & 0.015 \\ 
2002VE95    &  TNO &  0.009 & 0.006 \\ 
2002WC19    &  TNO &  0.008 & 0.002 \\ 
2003OP32    &  TNO &  0.013 & 0.013 \\ 
2003UZ413   &  TNO &  0.016 & 0.007 \\ 
2003VS2     &  TNO &  0.012 & 0.005 \\ 
2004NT33    &  TNO &  0.011 & 0.010 \\ 
2004PF115   &  TNO &  0.013 & 0.014 \\ 
2004SB60    &  TNO &  0.013 & 0.008 \\ 
2004TY364   &  TNO &  0.009 & 0.017 \\ 
2004UX10    &  TNO &  0.024 & 0.023 \\ 
2005QU182   &  TNO &  0.011 & 0.014 \\ 
2005RM43    &  TNO &  0.021 & 0.019 \\ 
2005RN43    &  TNO &  0.011 & 0.005 \\ 
2005RR43    &  TNO &  0.009 & 0.010 \\ 
2005UQ513   &  TNO &  0.019 & 0.021 \\ 
2007OR10    &  TNO &  0.017 & 0.011 \\ 
2007UK126   &  TNO &  0.009 & 0.009 \\ 
2008OG19    &  TNO &  0.022 & 0.010 \\ 
1997CU26(Chariklo) &  Cen &  0.007 & 0.007 \\ 
2000EB173   &  TNO &  0.007 & 0.006 \\ 
2002GB10 (Amycus)  &  Cen &  0.007 & 0.017 \\ 
2002GO9 (Crantor)    &  Cen &  0.009 & 0.007 \\ 
2002KX14    &  TNO &  0.012 & 0.010 \\ 
2002MS4     &  TNO &  0.009 & 0.005 \\ 
2003MW12    &  TNO &  0.006 & 0.006 \\ 
2007JJ43    &  TNO &  0.006 & 0.008 \\ 
1999DE9     &  TNO &  0.014 & 0.014 \\ 
2003FY128   &  TNO &  0.008 & 0.009 \\ 
2004GV9     &  TNO &  0.010 & 0.009 \\ 
2005CC79    &  Cen &  0.012 & 0.012 \\ 
2007JH43    &  TNO &  0.012 & 0.009 \\ 
2010EK139   &  TNO &  0.009 & 0.006 \\ 
IXION       &  TNO &  0.064 & 0.023 \\ 
ORCUS       &  TNO &  0.007 & 0.007 \\ 
QUAOAR      &  TNO &  0.013 & 0.014 \\ 
2003AZ84    &  TNO &  0.025 & 0.024 \\ 
\hline
\end{tabular}}
\label{table:incerteza}
\end{center}
\end{table}

%======================================================================
\subsection{Dificuldades encontradas e como elas estão sendo superadas}

A principal dificuldade centrou-se na utilização da ferramenta PRAIA, cujos
melhores resultados depende do ajuste de arquivo de parâmetros e, com frequência, da investigação visual de imagens.

%======================================================================

%======================================================================
\subsection{Bibliografia utilizada no contexto do trabalho}
%======================================================================
\noindent Fundamentals of Astrometry, Jean Kovalevsky, P. Kenneth Seidelmann, Cambridge University Press, 2004

\noindent M. Assafin, J. I. B. Camargo, R. Vieira Martins, et al. A\&A 515, A32 (2010)

\noindent J. I. B. Camargo, R. Vieira-Martins, M. Assafin, et al. A\&A 561, A37 (2014)

\noindent J. Desmars, J. I. B. Camargo, F. Braga-Ribas, et al. A\&A 584, A96 (2015) 

\noindent L. Lindegren, U. Lammers, U. Bastian, et al. A\&A 595, A4 (2016)
\vspace{-0.3 cm}
%===================================================================
\section{Proposta para as próximas etapas}
%===================================================================

\subsection{Atividades de pesquisa previstas}
% * <gugabrossi@gmail.com> 2017-11-24T12:53:54.449Z:
% 
% Acho que você pode tirar 'para o próximo período' do título das seções...
% 
% ^.
As atividades de pesquisa previstas para o pr\'oximo per\'iodo s\~ao:

\begin{enumerate}
\item Redução imagens obtidas durante o ano 2016 pelo telescópio 2,2 m do ESO.
\item Determinar a astrometria de Chariklo a partir da ocultação observada o dia 24 de Agosto de 2017. 
\item Obter os dados da segunda liberação do satélite GAIA que esta prevista para abril do 2018. Retomar as redu\c c\~oes.
\item Integrar os scripts utilizados na atualiza\c c\~ao do banco de oculta\c c\~oes estelares para obtermos um procedimento autom\'atico para tal tarefa.
\end{enumerate}
%------------------------------------------------------------------------------
\subsection{Atividades acadêmicas previstas para o próximo período}

As atividades acadêmicas previstas para o pr\'oximo per\'iodo s\~ao:

\begin{enumerate}[resume]
  \item Apresentar a pesquisa desenvolvida na Jornada de Seminários da PG-Astronomia.
  \item Exame de proficiência da L\'ingua Inglesa.
  \item Defesa da dissertação de mestrado.  
\end{enumerate}
%-------------------------------------------------------------------------------
\newpage
\subsection{Cronograma detalhado das atividades}
Segundo as propostas, em seguida se mostra uma tabela com o cronograma de atividades. Os números da primeira coluna corresponden aos itens das atividades  citadas acima.
\begin{table}[H]
\centering
\scalebox{1}{
\begin{tabular}{cccccccccccc}
\toprule \multirow{2}{*}{\textbf{Atividade}}& \multicolumn{1}{c}{\textbf{2017}} & \multicolumn{9}{c}{\textbf{2018}} \\ \cline{2-12}
  & \textbf{Dez} & \textbf{Jan} & \textbf{Fev} & \textbf{Mar} & \textbf{Abr} & \textbf{Mai} & \textbf{Jun} & \textbf{Jul} & \textbf{Ago} & \textbf{} & \textbf{}\\ \midrule 
1 & X & X &   &   &   &   &   &   &   &   \\ 
2 &   &   & X & X &   &   &   &   &   &   \\
3 &   &   &   &   & X & X & X & X &   &   \\ 
4 &   &   &   &   & X & X & X & X &   &   \\ 
5 &   &   &   & X &   &   &   &   &   &   \\
6 &   &   &   &   & X &   &   &   &   &   \\
7 &   &   &   &   &   &   &   &   & X &   \\ \bottomrule
\end{tabular}}
\caption{Cronograma das pr\'oximas etapas.}
\label{tab:cronograma}
\end{table}
\vspace{-0.8cm}
%------------------------------------------------------------------------------------ 

%==============================================================

\newpage
\bibliographystyle{unsrtnat}
\bibliography{bibliografia} 


\vfill \hfill
\begin{tabular}{p{5cm}p{5cm}p{5cm}}
\textsc{Julio I. Bueno de Camargo}  & \textsc{Gustavo Beneddeti Rossi}  \vspace{0.5cm} &  \textsc{Fredi Quispe Huaynasi}\\  
\end{tabular}

\end{document}
