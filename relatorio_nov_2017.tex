\documentclass[a4paper, 11pt]{article}
\usepackage{graphicx}
\usepackage{fourier}
\usepackage{anysize}
\usepackage{amsmath}
\usepackage{booktabs}          %Para ter mais opcoes nas tabelas
\usepackage[brazilian]{babel}
\usepackage[utf8]{inputenc}
\usepackage{float}
%\usepackage{pdfpages}
\usepackage{verbatim}
\usepackage[colorlinks]{hyperref}
\usepackage{multirow}
%\usepackage[comma]{natbib}
\usepackage{comment}
\usepackage[compress,comma,]{natbib}
\usepackage[bottom]{footmisc}    %footnote in bottom position

\marginsize{2.5cm}{2cm}{2cm}{2cm}


\renewcommand{\baselinestretch}{1.2}

\begin{document}
\renewcommand{\figurename}{\textsc{Figura}}
\renewcommand{\tablename}{\textsc{Tabela}}
\renewcommand{\refname}{Refer\^encias}

\numberwithin{table}{section}

\thispagestyle{empty}
{\large

{\scshape Observat\'orio Nacional

Minist\'erio da Ci\^encia, Tecnologia e Inova\c{c}\~ao - MCTI

Programa de P\'os-gradua\c{c}\~ao em astronomia}

\vspace{1cm}

\begin{center}
\includegraphics[width=5cm]{logoON.jpg} 
\end{center}

\vspace{1cm}

\begin{center}
\bfseries\Large Astrometria de TNOs e Centauros na era Gaia: procedimentos observacionais e imagens de época antiga
\end{center}

\vspace{1cm}

{\scshape
\begin{center}
Aluno: Fredi Quispe Huaynasi	

Orientador: Dr. Julio Ign\'acio Bueno de Camargo

Co-orientador: Dr. Gustavo Beneddeti Rossi
\end{center}

\vspace{1cm}

\begin{center}
Inicio de Mestrado: Setembro de 2016 

Bolsista da CNPq: Setembro de 2016
\end{center}}

\vfill

\begin{center}
Rio de Janeiro

Novembro de 2017 
\end{center}



}

\newpage


\section{Projeto}

\subsection{Objetivo científico}

A astrometria de objetos transnetunianos (TNOs) e Centauros pode ser determinada de duas formas: 1) tomando imagens diretas do corpo contendo estrelas em torno de eles (estrelas de referencia) e 2) atraves de ocultações estelares. Em qualquer dos casos, a astrometria depende da precisão astrométrica das estrelas de referencia. Neste sentido, os dados obtidos pela missão astrométrica GAIA (Global Astrometric Instrument for Astrophysics), a qual fornecera a astrometria para aproximadamente um bilhão de estrelas com magnitudes até 20 com uma precisão da ordem dos micro segundos ($\mu$as), permitira melhorar significativamente a precisão astrométrica dos TNOs e Centauros.

Conforme mencionado anteriormente, o objetivo cientifico do projeto de mestrado é, \textbf{usando a solução astrométrica GAIA contemplando os 5 parâmetros - posições, paralaxes, e movimentos próprios - a qual será liberada em Junho de 2017, determinar a astrometría de TNOs e Centauros usando imagens recentes e de época antiga, e atraves de ocultações estelares}.

\subsection{Estado da arte}

A técnica de ocultação estelar por TNOs e Centauros, complementada por outras técnicas, permite obter o tamanho de esses corpos na ordem do quilômetro, caracterizar sua forma, e revelar propiedades muito interessantes como anéis \citep{2017Natur.550..219O, 2014AGUFM.P43F..01B} e atmosferas em torno de eles \citep{1538-3881-136-5-1757, 2006Natur.439...52S}. No entanto, devido a que os TNOs e Centauros são pequenos (menores que 2400 km) e dada suas distancias a partir do Sol (entre  5 - 30 ua para os centauros, e maior que 30 ua para os TNOs) possuem baixo brilho (magnitudes maiores que 20), a previsão de ocultações por esses objetos não é uma tarefa simples. 

Para a predição de ocultações estelares é necessário conhecer a astrometria do corpo e da estrela que será ocultada. A astrometria do corpo é determinada a partir de imagens usando estrelas de referencia e a astrometria da estrela ocultada e das estrelas de referencia são tiradas de catálogos estelares. No passar do tempo, a precisão na posição das estrelas dos catálogos foi acrescentando com o uso de telescópios de grande porte. Não obstante, a precisão aumento consideravelmente com a missão espacial astrométrica Hipparcos produzindo um mapeamento de 120 000 estrelas no ordem de milissegundos. Devido ao exito da missão Hipparcos, foi lançada a missão GAIA, a qual permitira medir a posição das estrelas no ordem de microssegundos.

Esses resultados permitiram ter uma precisão maior na solução astrométrica dos TNOs e Centauros, e por conseguinte, melhorar significativamente a predição de ocultações.
\newpage     
\subsection{Metodologia}

A metodologia para conseguir nosso objetivo será dividida em duas partes:

\subsubsection{Metodologia para obter a astrometria usando imagens}
\begin{enumerate}
  \item Revisitar a base de imagens recentes e época antiga dos TNOs e Centauros que foram obtidas com a câmera WFI (Wide Field Manager) instalada no telescópio de 2,2m do ESO.
  \item Processar imagens utilizando o pacote esowfi no IRAF (Image Reduction And Analysis Facility) para aquelas que ainda não foram processadas.
  \item Determinar a astrometria dos corpos utilizando o pacote PRAIA (Platform for Reduction of Astronomical Images Automatically) usando os dados do catalogo GAIA.
\end{enumerate}
\subsubsection{Metodologia para obter a astrometria atraves de ocultações estelares}
\begin{enumerate}
  \item Analisar as curvas de luz de ocultações estelares observadas.
  \item Ajustar a forma do corpo às curvas de luz.
  \item Determinar a posição do centro do objeto.
  \item Obter a astrometria do corpo usando a posição da estrela ocultada.
\end{enumerate}
\subsection{Resultados esperados}

Depois de aplicar as metodologias descritas acima nós esperamos ter a astrometria de TNOs e Centauros usando a informação mais precisa das estrelas do catalogo que será fornecido pelo satélite GAIA:

\section{Informa\c{c}\~oes curriculares}

\subsection{Disciplinas cursadas}

\begin{center}
\begin{tabular}{clccc}
\toprule
\textbf{N}&$\hspace{2cm}$\textbf{Disciplina} &\textbf{Ano/Semestre}&\textbf{Conceito}& \textbf{Cr\'editos} \\ \midrule
1 & Astronomia observacional                                             & 2016/2 & A & 04 \\ 
2 & Astroestatística                                                     & 2016/2 & A & 02 \\ 
3 & Astrofísica Planetária                                               & 2016/2 & B & 04 \\ 
4 & Tópicos em Formação e Evolução Primordial do Sistema Solar           & 2016/2 & B & 02 \\
5 & Astronomia de posição                                                & 2017/1 & A & 04 \\ 
6 & Astronomia Dinâmica                                                  & 2017/1 & A & 04 \\ 
7 & Métodos Númericos em Astronomia Dinâmica                             & 2017/1 & A & 04 \\ \midrule 
\multicolumn{4}{c}{\textbf{Total de cr\'editos obtidos}}               & \textbf{24} \\ \bottomrule
\end{tabular}
\end{center}

\subsection{Cursos e reuni\~oes cient\'ificas}

\begin{itemize}
	\item Participação no XXI Ciclo de Cursos Especiais ON/MCT, 2016, nos seguintes temas: 

  Computational Cosmology: Simulating Cosmic Structures, Prof. August Evrard (University of Michigan, US); \\
  Comets: Origin, Evolution and Interrelations, Prof. Julio Fernández (University of the Republic, Uruguay); \\
  Advanced stellar evolution: basic properties, nucleosynthesis, and final fates, Prof. Paola Marigo (University of Padova, Italia).

	\item Participação na Second Astrobiology school at Observatório Nacional. 2016, realizado no Rio de Janeiro, ON-MCTI, nos seguintes temas: 

  Star Formation and Planet Formation and Habitability: From the Solar neighborhood to the Multiverse, Prof. Fred C. Adams (University of Michigan, USA); \\ The Pathway to Earth 2.0: Discovery and Characterization of the Nearest Exo-Earths, Prof. James S. Jenkins (Universidad de Chile); \\
  Dust in Protoplanetary Disks: The first Steps of Planet Formation, Prof. Isa Oliveira (Observatorio Nacional, Brasil); \\
  Degradation of amino acids by cosmic radiation, Prof. Enio Frota da silveira; \\
  Kepler Mission: Planet Candidates Characteristics and Earth Analogs Ocurrences Rates, Prof. Eduardo Seperuelo Duarte (UFRJ); \\
  Energetic Processing of Materials Relevant to Astrobiology, Prof. Daniele Fulvio (PUC do Rio de Janeiro)
	

	\item Participação no XXII Ciclo de Cursos Especiais ON/MCT, 2017, nos seguintes temas: 

  Extrasolar Planets Around Nearby Stars, Dr. R. Paul Butler (Carnegie Institution of Washington, EUA); \\
  Observing the Formation and Evolution of Galaxies over 13 Billion Years, Dr. Christopher Conselice (University of Nottingham, Reino Unido); \\
  Impacts in the Solar System, Dr. Patrick Michel (Observatoire de la Cote d' Azur, França); \\
  The Theory and Applications of the Baryon Acoustic Oscillations, Dr. Nikhil Padmanabhan (Yale University, EUA).
\end{itemize}

\subsection{Exame de profici\^encia em L\'ingua Inglesa}

O exame de proficiência será feita no dia 5 de Abril de 2018.

\subsection{Outras atividades}

Alem das atividades para cumprir com o objetivo cientifico, participe das seguintes atividades:

\begin{enumerate}
  \item Participação na construção do site \url{http://www.linea.gov.br/020-data-center/acesso-a-dados-3/tno-dados/}, a qual fornece predição de ocultações para TNOs e Centauros. 
  \item Participação em campanhas de observação 
  \begin{itemize}
    \item Julho (noite 27 e 28)
    \item Setembro (noites 19 e 20)
    \item Outubro (noites 21 e 22)
  \end{itemize}
\end{enumerate}
%==============================================================
\section{Situação do projeto de pesquisa à época do último relatório}
%==============================================================
Este é o meu primeiro relatório como aluno do DPPG.
%============================================================================
\section{Descrição detalhada do trabalho de pesquisa desenvolvido no período}
%============================================================================

%===================================================================
\subsection{Metodologia aplicada}
%===================================================================

%===================================================================
\subsection{Resultados parciais já obtidos}
%===================================================================
Foram determinadas a astrometria de 43 TNOs usando as imagens obtidas durante o ano 2013 pelo telescópio 2,2 m do ESO, e a primeira liberação de dados (Gaia DR1). A tabela que se segue mostra os resultados já obtidos.  
%======================================================================
\subsection{Dificuldades encontradas e como elas estão sendo superadas}
A principal dificuldade centrou-se na utilização da ferramenta PRAIA, a qual esta sendo superada pela ajuda do manual do pacote, e pela supervisão do orientador.  
%======================================================================

%======================================================================
\subsection{Bibliografia utilizada no contexto do trabalho}
%======================================================================

%===================================================================
\section{Pr\'oximas etapas}
%===================================================================

As atividades de pesquisa previstas para o pr\'oximo per\'iodo ser\~ao:

\begin{enumerate}
\item Processar imagens obtidas durante o ano 2016 pelo telescópio 2,2 m do ESO.
\item Determinar a astrometria usando o pacote PRAIA. 
\item Determinar a astrometria de Chariklo a partir da ocultação observada o dia 24 de Agosto de 2017. 
\item Obter os dados da segunda liberação do satélite GAIA que esta prevista para abril do 2018.
\item Reproduzir as metodologias utilizadas para a determinação da astrometria de TNOs e Centauros usando o catalogo de estrelas com dados da segunda liberação.
\end{enumerate}

\begin{table}[H]
\centering
\begin{tabular}{cccccccccccc}
\toprule \multirow{2}{*}{\textbf{Atividade}}& \multicolumn{8}{c}{\textbf{2016}} & \multicolumn{3}{c}{\textbf{2017}} \\ \cline{2-12}
  & \textbf{Abr} & \textbf{Mai} & \textbf{Jun} & \textbf{Jul} & \textbf{Ago} & \textbf{Out} & \textbf{Nov} & \textbf{Dez} & \textbf{Jan} & \textbf{Fev} & \textbf{Mar}\\ \midrule 
1 & X & X &   &   &   &   &   &   &   &   &   \\ 
2 &   &   & X &   &   &   &   &   &   &   &   \\
3 &   &   & X & X &   &   &   &   &   &   &   \\ 
4 &   &   &   &   & X & X &   &   &   &   &   \\
5 &   &   &   &   &   &   & X & X & X & X &   \\
6 &   &   &   &   &   &   &   &   & X & X & X \\ \bottomrule
\end{tabular}
\caption{Cronograma das pr\'oximas etapas.}
\label{tab:cronograma}
\end{table}
 
\newpage
\bibliographystyle{plainnat}
\bibliography{bibliografia} 
\vfill \hfill
\begin{tabular}{c}

\\ \hline
\textsc{Fredi Quispe H.}
\end{tabular}

\end{document}
